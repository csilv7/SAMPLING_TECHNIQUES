\usepackage{fancyhdr} % Personalização de cabeçalhos e rodapés.
\setlength{\headheight}{13.59999pt} % Define a altura do cabeçalho.
\usepackage[utf8]{inputenc} % Codificação UTF-8 para caracteres especiais.
\usepackage[brazil]{babel} % Configurações para o português do Brasil.
\usepackage{graphicx} % Inclusão de gráficos e imagens.
\usepackage{latexsym,amssymb,amsmath,amsfonts} % Símbolos e fontes da AMS para matemática.
\usepackage{indentfirst} % Indenta o primeiro parágrafo de cada seção.
\usepackage{url} % Formatação correta de URLs.
\usepackage{enumerate} % Personalização de listas enumeradas.
\usepackage[ocgcolorlinks]{hyperref} % Links clicáveis com cores que não afetam a impressão.

\usepackage{caption} % Permite um melhor controle sobre as legendas
\usepackage{longtable} % Certifique-se de que o pacote longtable está incluído

\usepackage{float} % Para definir novos ambientes flutuantes.

% Configurações de estilo para cabeçalhos e rodapés
\pagestyle{fancy}
\setcounter{page}{1}
\renewcommand{\thefootnote}{\dagger}

% Cabeçalho
\lhead{\nouppercase{\leftmark}}  % Exibe a seção atual à esquerda
\chead{}  % Deixe o centro do cabeçalho vazio
\rhead{\thepage}  % Coloca o número da página no lado direito do cabeçalho

% Rodapé
\lfoot{\small SILVA, B. C. R.}  % Texto do lado esquerdo do rodapé
\cfoot{}  % Deixe o centro do rodapé vazio
\rfoot{\small Técnicas de Amostragem}  % Texto do lado direito do rodapé
\renewcommand{\footrulewidth}{0.4pt} % Adiciona uma linha horizontal no rodapé

\counterwithin{table}{section}
\counterwithin{figure}{section}
\counterwithin{equation}{section}

% Definir novo ambiente "quadro" como um ambiente flutuante do tipo figura
\newfloat{quadro}{H}{qlo} % 'qlo' é uma extensão de lista de quadros
\floatname{quadro}{Quadro} % Define o nome do novo ambiente como "Quadro"

\captionsetup[quadro]{labelformat=default, labelsep=colon} % Formato da legenda para o ambiente "quadro"
\counterwithin{quadro}{section} % Numeração dos quadros por seção